\documentclass[../Main.tex]{subfiles}
\begin{document}
  \chapter{2021-2022学年微积分(一)(上)期中考试}

  \section{基本计算题(每小题 6 分,共 60 分)}
  \begin{enumerate}
    \item 求极限$\lim_{n\to +\infty}\left(\frac{a^{n}+b^{n}+c^{n}}{3}\right)^{\frac{1}{n}}
      (a,b,c>0)$.
      \vspace{11em}

    \item 求当$x\to0^{+}$时,无穷小量$\sqrt{1+x\sqrt{x}}-\mathrm{e}^{2x}$的主部和阶数.
      \vspace{11em}

    \item 求极限$l=\lim_{x\to0}\frac{1}{x^{3}}\left[\left(\frac{3+\cos x}{4}\right
      )^{x}-1\right]$.
      \vspace{11em}

    \item 已知$\lim_{x\to+\infty}\left(\sqrt{2x^{2}+4x-1}-ax-b\right)=0$,求$a,b$.
      \vspace{11em}

    \item 求极限$l=\lim_{x\to0}\left(\frac{\ln(1+x)}{x}\right)^{\frac{1}{\mathrm{e}^{x}-1}}$.
      \vspace{9em}

    \item 确定$f(x)=\mathrm{e}^{\frac{|x|}{\tan x}}$的间断点及其类型.
      \vspace{10em}

    \item 设函数$y=y(x)$由参数方程$\begin{cases}
        x=t+\mathrm{e}^{t}, \\
        y=\sin t
      \end{cases}$确定,求$\left.\frac{\mathrm{d}^{2} y}{\mathrm{d}x^{2}}\right|_{t=0}$.
      \vspace{10em}

    \item 设$y=x^{x^x}$,求$y'$.
      \vspace{10em}

    \item 求$y=x^{2}\ln(1+x)$在$x=0$处的$n$阶导数.
      \vspace{8em}

    \item 设$y=\frac{\cos x}{x^{2}}$,求$\frac{\mathrm{d}y}{\mathrm{d}(\cos x)}$,$\frac{\mathrm{d}y}{\mathrm{d}(x^{3})}$.
      \vspace{8em}
  \end{enumerate}

  \section{综合题(每小题 6 分,共 30 分)}
  \begin{enumerate}
    \item[11.] 设$x=\varphi(y)$是$f(x)=\ln x+\arctan x$的反函数,求$\varphi'\left
      (\frac{\pi}{4}\right)$.
      \vspace{10em}

    \item[12.] 设曲线$y=y(x)$由方程$\mathrm{e}^{xy}+\ln\frac{y}{x+1}=2$确定,求曲线在$x
      =0$处的切线方程.
      \vspace{10em}

    \item[13.] 设函数$f(x)$在$x=0$处可导,且$f(0)=0$,$f'(0)=2$,计算$\lim_{x\to0}
      \frac{x^{2}f(x)-2f(x^{3})}{x^{3}}$.
      \vspace{10em}

    \item[14.] 有一长度为$5\mathrm{m}$的梯子贴靠在铅直的墙上,假设其下端沿地板离开墙角而滑动.当梯子下端离开墙角$3
      \mathrm{m}$时,已知梯子的下端离开墙角滑动速率为$2.2\mathrm{m/s}$,问此时梯子的上端向下滑的速率为多少?
      \vspace{10em}

    \item[15.] 设函数$f(x)=
      \begin{cases}
        x^{2}\cos\frac{1}{x}, & x\neq0, \\
        0,                    & x=0
      \end{cases}$,求$f'(x)$,并讨论$f'(x)$的连续性.
      \vspace{12em}
  \end{enumerate}

  \section{证明题(每小题 5 分,共 10 分)}
  \begin{enumerate}
    \item[16.] 设$x_{1}>0$,$x_{n+1}=3+\frac{4}{x_{n}}(n=1,2,\cdots)$,证明:$\lim
      _{n\to\infty}x_{n}$存在并求其值.
      \vspace{12em}

    \item[17.] 设函数$f(x)$在$[0,1]$上连续,$(0,1)$内可导,$c\in(0,1)$.证明:$\exists
      \,\xi,\eta\in[0,1]$,使得
      \[
        2\eta f(1)+(c^{2}-1)f'(\eta)=f(\xi).
      \]
      \vspace{10em}
  \end{enumerate}

  \chapter{2021-2022学年微积分(一)(上)期中考试参考答案}
  \section{基本计算题(每小题 6 分,共 60 分)}
  \begin{enumerate}
    \item \textbf{Solution}. 令$A=\max\{a,b,c\}$,则
      \[
        \frac{1}{3}A^{n}\leq\frac{a^{n}+b^{n}+c^{n}}{3}\leq A^{n},\quad \frac{1}{\sqrt[n]{3}}
        A\leq\left(\frac{a^{n}+b^{n}+c^{n}}{3}\right)^{\frac{1}{n}}\leq A.
      \]

      由夹逼准则知,原式$=A=\max\{a,b,c\}$.

    \item \textbf{Solution}.
      \[
        \begin{aligned}
          f(x)= & \sqrt{1+x\sqrt{x}}-\mathrm{e}^{2x}=\left(\sqrt{1+x\sqrt{x}}-1\right)-\left(\mathrm{e}^{2x}-1\right)(x\to 0^{+}) \\
          =     & \frac{1}{2}x\sqrt{x}+o(x)-(2x+o(x))                                                                             \\
          =     & -2x+o(x)\sim -2x(x\to0^{+}).
        \end{aligned}
      \]

      故主部为$-2x$,阶数为$1$.

    \item \textbf{Solution}.

      \[
        \begin{aligned}
          l= & \lim_{x\to0}\frac{\mathrm{e}^{x\ln\frac{3+\cos x}{4}}-1}{x^3}                               \\
          =  & \lim_{x\to0}\frac{x\ln\frac{3+\cos x}{4}}{x^3}=\lim_{x\to0}\frac{\frac{3+\cos x}{4}-1}{x^2} \\
          =  & \frac{1}{4}\lim_{x\to0}\frac{\cos x-1}{x^2}=-\frac{1}{8}.
        \end{aligned}
      \]

    \item \textbf{Solution}. 要使$\lim_{x\to+\infty}\left(\sqrt{2x^{2}+4x-1}-ax-b
      \right)=0$成立,必须
      \[
        \lim_{x\to+\infty}\left(\frac{\sqrt{2x^{2}+4x-1}}{x}-a-\frac{b}{x}\right)
        =0,
      \]

      从而得
      \[
        a=\lim_{x\to+\infty}\frac{\sqrt{2x^{2}+4x-1}}{x}=\lim_{x\to+\infty}\sqrt{2+\frac{4}{x}-\frac{1}{x^{2}}}
        =\sqrt{2}.
      \]

      \[
        \begin{aligned}
          b = & \lim_{x\to+\infty}\left(\sqrt{2x^2+4x-1}-ax\right)=\lim_{x\to\infty}\left(\sqrt{2x^2+4x-1}-\sqrt{2}x\right) \\
          =   & \lim_{x\to+\infty}\sqrt{2}x\left[\sqrt{1+\left(\frac{2}{x}-\frac{1}{2x^2}\right)}-1\right]                  \\
          =   & \lim_{x\to+\infty}\sqrt{2}x\cdot\frac{1}{2}\left(\frac{2}{x}-\frac{1}{2x^2}\right)=\sqrt{2}.
        \end{aligned}
      \]

    \item \textbf{Solution}. $l=\lim_{x\to0}\mathrm{e}^{\frac{1}{\mathrm{e}^{x}-1}\ln\frac{\ln(1+x)}{x}}
      =\mathrm{e}^{\lim_{x\to0}\frac{1}{\mathrm{e}^{x}-1}\ln\frac{\ln(1+x)}{x}}$.

      而
      \[
        \begin{aligned}
          \lim_{x\to0}\frac{1}{\mathrm{e}^x-1}\ln\frac{\ln(1+x)}{x}= & \lim_{x\to0}\frac{1}{\mathrm{e}^x-1}\left[\frac{\ln(1+x)}{x}-1\right] \\
          =                                                          & \lim_{x\to}\frac{\ln(1+x)-x}{x^2}                                     \\
          =                                                          & \lim_{x\to0}\frac{\frac{1}{1+x}-1}{2x}                                \\
          =                                                          & \lim_{x\to0}\frac{1}{1+x}\cdot\frac{1-1-x}{2x}=-\frac{1}{2}.
        \end{aligned}
      \]

      所以$l=\mathrm{e}^{-\frac{1}{2}}$.

    \item \textbf{Solution}. $f(x)$的间断点为$x=k\pi$,$x=k\pi+\frac{\pi}{2}$,$k
      =0,\pm1,\pm2,\cdots$.

      $x=0$是跳跃间断点.因为
      \[
        \lim_{x\to0^+}\mathrm{e}^{\frac{|x|}{\tan x}}=\mathrm{e}^{\lim_{x\to0^+}\frac{x}{\tan
        x}}=\mathrm{e},\quad \lim_{x\to0^-}\mathrm{e}^{\frac{|x|}{\tan x}}=\mathrm{e}
        ^{\lim_{x\to0^-}\frac{-x}{\tan x}}=\mathrm{e}^{-1}.
      \]

      $x=k\pi(k=\pm1,\pm2,\cdots)$是无穷间断点.因为
      \[
        \lim_{x\to k\pi^+}\mathrm{e}^{\frac{|x|}{\tan x}}=\mathrm{e}^{+\infty}=+\infty
        .
      \]

      $x=k\pi+\frac{\pi}{2}(k=0,\pm1,\pm2,\cdots)$是可去间断点.因为
      \[
        \lim_{x\to k\pi+\frac{\pi}{2}}\mathrm{e}^{\frac{|x|}{\tan x}}=\mathrm{e}^{0}
        =1.
      \]

    \item \textbf{Solution}. $\frac{\mathrm{d}y}{\mathrm{d}x}=\frac{\frac{\mathrm{d}y}{\mathrm{d}t}}{\frac{\mathrm{d}x}{\mathrm{d}t}}
      =\frac{\cos t}{1+\mathrm{e}^{t}}$,

      $\frac{\mathrm{d}^{2} y}{\mathrm{d}x^{2}}=\frac{\frac{-\sin t\left(1+\mathrm{e}^{t}\right)-\cos
      t\cdot\mathrm{e}^{t}}{\left(1+\mathrm{e}^{t}\right)^{2}}}{1+\mathrm{e}^{t}}
      =\frac{-\sin t\left(1+\mathrm{e}^{t}\right)-\cos t\cdot\mathrm{e}^{t}}{\left(1+\mathrm{e}^{t}\right)^{3}}$,

      $\left.\frac{\mathrm{d}^{2} y}{\mathrm{d}x^{2}}\right|_{t=0}=-\frac{1}{8}$.

    \item \textbf{Solution}. 因$\left(x^{x}\right)'=\left(\mathrm{e}^{x\ln x}\right
      )'=x^{x}(1+\ln x)$,又
      \[
        \ln y=x^{x}\ln x,
      \]

      所以$\frac{1}{y}y'=\left(x^{x}\ln x\right)'=x^{x}(1+\ln x)\ln x+x^{x-1}$,即
      \[
        y'=x^{x^x}x^{x}\left(\frac{1}{x}+\ln x+\ln^{2} x\right).
      \]

    \item \textbf{Solution}. 取$v(x)=x^{2}$,它的三阶以上的导数为0,
      \[
        u^{(k)}(x)=\left[\ln(1+x)\right]^{(k)}=\frac{(-1)^{k-1}(k-1)!}{(1+x)^{k}}
        ,k=1,2,\cdots,
      \]

      由Leibniz公式$(uv)^{(n)}=\sum_{k=0}^{n}\mathrm{C}_{n}^{k}u^{(n-k)}v^{(k)}$,得
      \[
        y^{(n)}=x^{2}\frac{(-1)^{n-1}(n-1)!}{(1+x)^{n}}+2nx\frac{(-1)^{n-2}(n-2)!}{(1+x)^{n-1}}
        +n(n-1)\frac{(-1)^{n-3}(n-3)!}{(1+x)^{n-2}}.
      \]

      所以$y^{(n)}(0)=(-1)^{n-3}n(n-1)(n-3)!=\frac{(-1)^{n}n!}{n-2}(n>2)$.

      而$y'(0)=y''(0)=0$.

    \item \textbf{Solution}. $\frac{\mathrm{d}y}{\mathrm{d}x}=\frac{-x^{2}\sin x-2x\cos
      x}{x^{4}}=-\frac{x\sin x+2\cos x}{x^{3}}$.

      $\frac{\mathrm{d}y}{\mathrm{d}(\cos x)}=\frac{-\frac{x\sin x+2\cos x}{x^{3}}\mathrm{d}x}{-\sin
      x\mathrm{d}x}=\frac{x+2\cot x}{x^{3}}$.

      $\frac{\mathrm{d}y}{\mathrm{d}(x^{3})}=\frac{-\frac{x\sin x+2\cos x}{x^{3}}\mathrm{d}x}{3x^{2}\mathrm{d}x}
      =-\frac{x\sin x+2\cos x}{3x^{5}}$.
  \end{enumerate}

  \section{综合题(每小题 6 分,共 30 分)}
  \begin{enumerate}
    \item[11.] \textbf{Solution}. 当$x=1$时,$y=f(1)=\frac{\pi}{4}$.

      $f'(x)=\left(\ln x+\arctan x\right)'=\frac{1}{x}+\frac{1}{1+x^{2}}$,

      得$f'(1)=1+\frac{1}{2}=\frac{3}{2}$,故$\varphi'\left(\frac{\pi}{4}\right)=
      \frac{1}{f'(1)}=\frac{2}{3}$.

    \item[12.] \textbf{Solution}. 原方程变为$\mathrm{e}^{xy}(y+xy')+\frac{y'}{y}-
      \frac{1}{x+1}=0$,

      将$x=0,y=\mathrm{e}$代入,得$\mathrm{e}+\frac{y'(0)}{\mathrm{e}}=1$,故$y'(
      0)=\mathrm{e}(1-\mathrm{e})$.

      所以切线方程为$y=\mathrm{e}(1-\mathrm{e})x+\mathrm{e}$.

    \item[13.] \textbf{Solution}. $y'(x)=2f(x)f'(x)$,

      \[
        \begin{aligned}
          \lim_{x\to0}\frac{x^2f(x)-2f(x^3)}{x^3}= & \lim_{x\to0}\left[\frac{f(x)-f(0)}{x}-2\frac{f(x^3)-f(0)}{x^3}\right] \\
          =                                        & f'(0)-2f'(0)=-f'(0)                                                   \\
          =                                        & -2.
        \end{aligned}
      \]

    \item[14.] \textbf{Solution}. 设梯子上端离墙角距离为$s(m)$,下端离开墙角的距离为$x
      (m)$,有一长度为
      \[
        s=\sqrt{5^{2}-x^{2}}.
      \]

      于是$\frac{\mathrm{d}s}{\mathrm{d}t}=-\frac{x}{\sqrt{25-x^{2}}}\frac{\mathrm{d}x}{\mathrm{d}t}$.

      当$x=3\mathrm{m}$,$\frac{\mathrm{d}x}{\mathrm{d}t}=2.2\mathrm{m/s}$时,$\frac{\mathrm{d}s}{\mathrm{d}t}
      =-\frac{3}{4}\cdot2.2=-1.65\mathrm{m/s}$.

      即梯子上端向下滑的速率为$1.65\mathrm{m/s}$.

    \item[15.] \textbf{Solution}. 因$f'(0)=\lim_{x\to0}\frac{f(x)-f(0)}{x}=\lim_{x\to0}
      x\cos\frac{1}{x}=0$,

      且$x\neq0$时,$f'(x)=2x\cos\frac{1}{x}+\sin\frac{1}{x}$,所以
      \[
        f(x)=
        \begin{cases}
          2x\cos\frac{1}{x}+\sin\frac{1}{x}, & x\neq 0, \\
          0,                                 & x=0.
        \end{cases}
      \]

      当$x\neq0$时,初等函数$f'(x)=2x\cos\frac{1}{x}+\sin\frac{1}{x}$有定义,所以连续;

      而$\lim_{x\to0}\left(2x\cos\frac{1}{x}+\sin\frac{1}{x}\right)$不存在,所以$f
      '(x)$在$x=0$处不连续.
  \end{enumerate}

  \section{证明题(每小题 5 分,共 10 分)}
  \begin{enumerate}
    \item[16.] \textbf{Proof}.

      \textbf{法一.} 由题设知$n>1$时,$x_{n}>3$;$n>2$时,$x_{n}<3+\frac{4}{3}$,即$\{
      x_{n}\}$有界.

      由$x_{n+1}-x_{n}=\frac{4}{x_{n}}-\frac{4}{x_{n-1}}=\frac{4(x_{n-1}-x_{n})}{x_{n}
      x_{n-1}}$知不能确定$\{x_{n}\}$的单调性,但由
      \[
        x_{n+1}-x_{n-1}=\frac{4}{x_{n}}-\frac{4}{x_{n-2}}=\frac{4(x_{n-2}-x_{n})}{x_{n}
        x_{n-2}}=\frac{16(x_{n-1}-x_{n-3})}{x_{n} x_{n-1}x_{n-2}x_{n-3}}
      \]

      知奇子列$\{x_{2k-1}\}$与偶子列$\{x_{2k}\}$均分别单调.由单调有界原理知奇子列$\{
      x_{2k-1}\}$与偶子列$\{x_{2k}\}$均收敛.

      设$\lim_{k\to\infty}x_{2k-1}=l_{1}$,$\lim_{k\to\infty}x_{2k}=l_{2}$,

      在
      \[
        x_{2k+1}=3+\frac{4}{x_{2k}},\quad x_{2k}=3+\frac{4}{x_{2k-1}}
      \]

      两边取极限,得$l_{1}=3+\frac{4}{l_{2}}$以及$l_{2}=3+\frac{4}{l_{1}}$,解得$l
      _{1}=l_{2}=4$.

      因此数列$\{x_{n}\}$的极限存在,且$\lim_{n\to\infty}x_{n}=4$.

      \textbf{法二.} 常数$l=4$满足$l=3+\frac{4}{l}$.下证数列$\{x_{n}\}$以$l$为极限.

      \[
        \begin{aligned}
          0\leq|x_{n+1}-l|= & \left|\left(3+\frac{4}{x_n}\right)-4\right|=\frac{|x_n-4|}{x_n} \\
          \leq              & \frac{1}{3}|x_{n}-4|                                            \\
          \leq              & \cdots \leq \frac{1}{3^{n-1}}|x_{2}-4|.
        \end{aligned}
      \]

      由迫敛性知$|x_{n}-l|$收敛到$0$,故数列$\{x_{n}\}$以$l$为极限.

    \item[17.] \textbf{Proof}. 构造函数$F(x)=x^{2}f(1)+(c^{2}-1)f(x)$,则$F(x)$在$[
      0,1]$上连续,$(0,1)$内可导.

      由Lagrange中值定理,$\exists\,\eta\in(0,1)$,使得
      \[
        F(1)-F(0)=F'(\eta),
      \]

      即$2\eta f(1)+(c^{2}-1)f'(\eta)=(1-c^{2})f(0)+c^{2}f(1)$,

      因$(1-c^{2})f(0)+c^{2} f(1)$是$f(0)$与$f(1)$的加权平均值,由介值定理知,$\exists
      \,\xi\in[0,1]$,使得
      \[
        f(\xi)=(1-c^{2})f(0)+c^{2} f(1),
      \]

      故$\exists\,\xi,\eta\in[0,1]$,使得
      \[
        2\eta f(1)+(c^{2}-1)f'(\eta)=f(\xi).
      \]
  \end{enumerate}
\end{document}