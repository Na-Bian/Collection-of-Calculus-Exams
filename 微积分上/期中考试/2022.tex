\documentclass[../Main.tex]{subfiles}
\begin{document}
  \chapter{2022-2023学年微积分(一)(上)期中考试}

  \section{基本计算题(每小题 6 分,共 60 分)}
  \begin{enumerate}
    \item 求极限$l=\lim_{n\to\infty}\left(\sin\frac{\pi}{\sqrt{n^{2}+1}}+\sin\frac{\pi}{\sqrt{n^{2}+2}}
      +\cdots+\sin\frac{\pi}{\sqrt{n^{2}+n}}\right)$.
      \vspace{11em}

    \item 求极限$l=\lim_{x\to0}\frac{1}{\ln(1+x)}\left(\frac{1}{x}-\frac{1}{\sin
      x}\right)$.
      \vspace{11em}

    \item 求当$x\to 0$时,无穷小量$x-\arctan x$的主部和阶数.
      \vspace{11em}

    \item 设$b=\lim_{x\to0}\frac{x+a\cos x-2}{\tan x}$为常数,求$a,b$.
      \vspace{11em}

    \item 设函数$f(x)=\frac{\ln |x|}{|x-1|}\sin x$,求$f(x)$的间断点并判断类型.
      \vspace{9em}

    \item 设$y(x)=\frac{\sqrt{1+x}-\sqrt{1-x}}{\sqrt{1+x}+\sqrt{1-x}}(x\neq0,x\neq
      \pm1)$,求$y'(x)$.
      \vspace{10em}

    \item 设二阶可导函数$y=y(x)$由方程$y=1+x\mathrm{e}^{y}$确定,求$\left.\frac{\mathrm{d}^{2}
      y}{\mathrm{d}x^{2}}\right|_{x=0}$.
      \vspace{10em}

    \item 设$y=(1+x^{2})^{\sin x}$,求$\mathrm{d}y$.
      \vspace{10em}

    \item 已知$f(x)=x^{3}\ln(1+x)$,求$f^{(10)}(0)$.
      \vspace{8em}

    \item 求曲线$r=1-\cos \theta$在$\theta=\frac{\pi}{2}$处的切线方程.
      \vspace{8em}
  \end{enumerate}

  \section{综合题(每小题 6 分,共 30 分)}
  \begin{enumerate}
    \item[11.] 研究函数$f(x)=\lim_{x\to\infty}\frac{x+x^{2}\mathrm{e}^{nx}}{1+\mathrm{e}^{nx}}$的连续性.
      \vspace{10em}

    \item[12.] 设$y=g(x)$是$y=f(x)$的反函数,$f(x)$可导,$f(1)=2,f'(1)=-4$,求$y=
      g(1+x^{2})$在$x=1$处的导数.
      \vspace{10em}

    \item[13.] 设函数$f(x)$在$x=1$处二阶可导,且$f'(1)=0$,$f''(1)=0$,$y=f^{2}(x
      )$,求$\left.\frac{\mathrm{d}^{2} y}{\mathrm{d}x^{2}}\right|_{x=1}$.
      \vspace{8em}

    \item[14.] 设函数$g(x)=
      \begin{cases}
        x^{3}\sin\frac{1}{x}, & x\neq0, \\
        0,                    & x=0
      \end{cases}$,函数$f(x)$可导,求$F(x)=f(g(x))$的导数.
      \vspace{10em}

    \item[15.] 将水以$4\mathrm{m}^{3}/\mathrm{min}$的速率注入一个圆锥形容器中,容器顶朝下倒立,它的高度为$8
      \mathrm{m}$,底面半径为$4\mathrm{m}$,当容器内的水深达$5\mathrm{m}$时,水面升高的速率是多少?
      \vspace{12em}
  \end{enumerate}

  \section{证明题(每小题 5 分,共 10 分)}
  \begin{enumerate}
    \item[16.] 设$x_{n}>0$,$x_{n+1}+\frac{4}{x_{n}}<4(n=1,2,\cdots)$,证明:$\lim
      _{n\to\infty}x_{n}$存在并求其值.
      \vspace{12em}

    \item[17.] 设函数$f(x)$在$[0,b]$上具有二阶导数,且$|f''(x)|\leq M$,$f(x)$在$(
      0,b)$内取得最大值,试证:$|f'(0)|+|f'(b)|\leq Mb$.
      \vspace{10em}
  \end{enumerate}

  \chapter{2022-2023学年微积分(一)(上)期中考试参考答案}
  \section{基本计算题(每小题 6 分,共 60 分)}
  \begin{enumerate}
    \item \textbf{Solution}. 设$x_{n}=\sin\frac{\pi}{\sqrt{n^{2}+1}}+\sin\frac{\pi}{\sqrt{n^{2}+2}}
      +\cdots+\sin\frac{\pi}{\sqrt{n^{2}+n}}$,则
      \[
        n\sin\frac{\pi}{\sqrt{n^{2}+n}}\leq x_{n} \leq n\sin\frac{\pi}{\sqrt{n^{2}+1}}
        ,\quad n>1.
      \]

      而$\lim_{n\to\infty}n\sin\frac{\pi}{\sqrt{n^{2}+1}}=\lim_{n\to\infty}n\cdot
      \frac{\pi}{\sqrt{n^{2}+1}}=\pi$,$\lim_{n\to\infty}n\sin\frac{\pi}{\sqrt{n^{2}+n}}
      =\lim_{n\to\infty}n\cdot\frac{\pi}{\sqrt{n^{2}+n}}=\pi$,

      所以$l=\pi$.

    \item \textbf{Solution}.
      \[
        \begin{aligned}
          l= & \lim_{x\to 0}\frac{1}{x}\cdot\frac{\sin x-x}{x\sin x} \\
          =  & \lim_{x\to0}\frac{\sin x-x}{x^3}                      \\
          =  & \lim_{x\to0}\frac{\cos x-1}{3x^2}=-\frac{1}{6}.
        \end{aligned}
      \]

    \item \textbf{Solution}.

      \textbf{法一.} 设$f(x)=\arctan x$,则$f(0)=0$,$f'(0)=\left.\frac{1}{1+x^{2}}
      \right|_{x=0}=1$,

      $f''(0)=\left.-\frac{2x}{(1+x^{2})^{2}}\right|_{x=0}=0$,$f'''(0)=\left.\frac{-2(1+x^{2})^{2}+8x^{2}(1+x^{2})}{(1+x^{2})^{4}}
      \right|_{x=0}=-2$,

      所以$\arctan x=x-\frac{1}{3}x^{3}+o(x^{3})$,于是
      \[
        x-\arctan x=\frac{1}{3}x^{3}+o(x^{3})\sim\frac{1}{3}x^{3}.
      \]

      即主部为$\frac{1}{3}x^{3}$,阶数为$3$.

      \textbf{法二.} 设$x\to0$,$x-\arctan x$的主部为$cx^{r}$,则$\lim_{x\to0}\frac{x-\arctan
      x}{cx^{r}}=1$.

      因
      \[
        \begin{aligned}
          \lim_{x\to0}\frac{(x-\arctan x)'}{(cx^r)'}= & \frac{1-\frac{1}{1+x^2}}{crx^{r-1}} \\
          =                                           & \lim_{x\to0}\frac{1}{crx^{r-3}},
        \end{aligned}
      \]

      所以$\lim_{x\to0}\frac{1}{crx^{r-3}}=1$,从而$r=3,c=\frac{1}{3}$,即主部为$\frac{1}{3}
      x^{3}$,阶数为$3$.

    \item \textbf{Solution}. 由$b$是常数,$\lim_{x\to0}\tan x=0$知,$\lim_{x\to0}
      (x+a\cos x-2)=0$,于是$a=2$.

      进而$b=\lim_{x\to0}\frac{x+2(\cos x-1)}{x}=1$,即$a=2,b=1$.

    \item \textbf{Solution}. 当$x=0,x=1$时,$f(x)$无定义,故$x=0,x=1$是$f(x)$的间断点.

      因
      \[
        \begin{aligned}
          \lim_{x\to0^+}f(x)= & \lim_{x\to0^+}\frac{\ln x}{\csc x}\cdot\frac{1}{|x-1|} \\
          =                   & \lim_{x\to0^+}\frac{\frac{1}{x}}{-\csc x\cot x}        \\
          =                   & -\lim_{x\to0^+}\frac{\sin ^2 x}{x\cos x}=0.
        \end{aligned}
      \]

      类似可得$\lim_{x\to0^-}f(x)=0$,所以$x=0$是可去间断点.

      (或$\lim_{x\to0}f(x)=\lim_{x\to0}x\ln|x|\cdot\frac{1}{|x-1|}=\lim_{x\to 0}
      x\ln|x|=\lim_{x\to0}\frac{\ln|x|}{\frac{1}{x}}=\lim_{x\to0}\frac{\frac{1}{x}}{-\frac{1}{x^{2}}}
      =0$)

      又
      \[
        \begin{aligned}
          \lim_{x\to1^+}f(x)= & \lim_{x\to1^+}\frac{\ln x}{x-1}\sin x=\lim_{x\to1^+}\frac{x-1}{x-1}\sin x=\sin 1 \\
          \lim_{x\to1^-}f(x)= & \lim_{x\to1^-}\frac{-\ln x}{x-1}\sin x=-\sin 1,
        \end{aligned}
      \]

      所以$x=1$是跳跃间断点.

    \item \textbf{Solution}. 化简得$y(x)=\frac{1-\sqrt{1-x^{2}}}{x}$.

      所以$y'(x)=\frac{-\frac{2x}{2\sqrt{1-x^{2}}}\cdot x-(1-\sqrt{1-x^{2}})}{x^{2}}
      =\frac{1-\sqrt{1-x^{2}}}{x^{2}\sqrt{1-x^2}}$.

    \item \textbf{Solution}. 由题设有$x=0,y=1$,方程两边关于$x$求导得
      \[
        \begin{gathered}
          (1-x\mathrm{e}^y)y'=\mathrm{e}^y,\qquad\text{(*)} \\
          y'(0)=\mathrm{e}.
        \end{gathered}
      \]

      方程(*)两边再关于$x$求导得
      \[
        (1-x\mathrm{e}^{y})y''=2y'\mathrm{e}^{y}+x\cdot(y')^{2}\cdot\mathrm{e}^{y}
        ,
      \]

      所以
      \[
        y''(0)=2\mathrm{e}^{2},\quad\text{即}\,\left.\frac{\mathrm{d}^{2} y}{\mathrm{d}x^{2}}
        \right|_{x=0}=2\mathrm{e}^{2}.
      \]

    \item \textbf{Solution}. 因$\ln y=\sin x\ln(1+x^{2})$,所以

      \[
        \frac{1}{y}y'=\cos x\ln(1+x^{2})+\sin x\cdot\frac{2x}{1+x^{2}},
      \]

      即$y'=\left(\cos x\ln(1+x^{2})+\frac{2x\sin x}{1+x^{2}}\right)y=(1+x^{2})^{\sin
      x}\left(\cos x\ln(1+x^{2})+\frac{2x\sin x}{1+x^{2}}\right)$,

      因此$\mathrm{d}y=(1+x^{2})^{\sin x}\left(\cos x\ln(1+x^{2})+\frac{2x\sin x}{1+x^{2}}
      \right)\mathrm{d}x$.

    \item \textbf{Solution}. 取$v(x)=x^{3}$,它的四阶以上的导数为0,
      \[
        u^{(k)}(x)=\left(\ln(1+x)\right)^{(k)}=\frac{(-1)^{k-1}(k-1)!}{(1+x)^{k}}
        ,k=1,2,\cdots,
      \]

      由Leibniz公式$(uv)^{(n)}=\sum_{k=0}^{n}\mathrm{C}_{n}^{k}u^{(n-k)}v^{(k)}$,得
      \[
        f^{(10)}(x)=x^{3}\frac{(-1)^{9} 9!}{(1+x)^{10}}+30x^{2}\frac{(-1)^{8} 8!}{(1+x)^{9}}
        +3\times10\times9x\frac{(-1)^{7} 7!}{(1+x)^{8}}+10\times9\times8\frac{(-1)^{6}
        6!}{(1+x)^{7}}.
      \]

      所以$f^{(10)}(0)=\frac{10!}{7}$.

    \item \textbf{Solution}.曲线的参数方程为$\begin{cases}
        x=(1-\cos\theta)\cos\theta, \\
        y=(1-\cos\theta)\sin\theta
      \end{cases}$.于是得到
      \[
        \frac{\mathrm{d}y}{\mathrm{d}x}=\frac{\frac{\mathrm{d}y}{\mathrm{d}\theta}}{\frac{\mathrm{d}x}{\mathrm{d}\theta}}
        =\frac{\sin^{2}\theta+(1-\cos\theta)\cos\theta}{2\sin\theta\cos\theta-\sin\theta}
        .
      \]
      代入$\theta=\frac{\pi}{2}$得到$\left.\frac{\mathrm{d}y}{\mathrm{d}x}\right|
      _{\theta=\frac{\pi}{2}}=-1$,$x=0,y=1$,因此切线方程为$y=-x+1$.
  \end{enumerate}

  \section{综合题(每小题 6 分,共 30 分)}
  \begin{enumerate}
    \item[11.] \textbf{Solution}. 当$x>0$时,$\mathrm{e}^{nx}\to+\infty(n\to\infty
      )$,因此$\lim_{n\to\infty}\frac{x^{2}+x^{2}\mathrm{e}^{nx}}{1+\mathrm{e}^{nx}}
      =x^{2}$;

      当$x<0$时,$\mathrm{e}^{nx}\to0(n\to\infty)$,因此$\lim_{n\to\infty}\frac{x+x^{2}\mathrm{e}^{nx}}{1+\mathrm{e}^{nx}}
      =x$;

      当$x=0$时,$\lim_{n\to\infty}\frac{x+x^{2}\mathrm{e}^{nx}}{1+\mathrm{e}^{nx}}
      =0$,

      综上所述,$f(x)=
      \begin{cases}
        x^{2}, & x>0,    \\
        x,     & x\leq0.
      \end{cases}$

      当$x\neq0$时,$f(x)=x^{2}$及$f(x)=x$均为幂函数,连续;

      当$x=0$时,$\lim_{x\to0^+}f(x)=0=\lim_{x\to0^-}f(x)$,且$f(0)=0$,故$f(x)$在$x
      =0$处连续.

      因此$f(x)$在$(-\infty,+\infty)$上处处连续.

    \item[12.] \textbf{Solution}. 由题意得$g'(2)=\frac{1}{f'(1)}=-\frac{1}{4}$.

      所以
      \[
        y'(1)=\left.[2xg'(2)]\right|_{x=1}=-\frac{1}{2}.
      \]

    \item[13.] \textbf{Solution}. $y'(x)=2f(x)f'(x)$,

      \[
        \begin{aligned}
          \left.\frac{\mathrm{d}^2 y}{\mathrm{d}x^2}\right|_{x=1}= & \lim_{x\to1}\frac{y'(x)-y'(1)}{x-1}        \\
          =                                                        & \lim_{x\to1}\frac{2f(x)f'(x)}{x-1}         \\
          =                                                        & \lim_{x\to1}\frac{2f(x)(f'(x)-f'(1))}{x-1} \\
          =                                                        & 2f(1)f''(1)=0.
        \end{aligned}
      \]

    \item[14.] \textbf{Solution}. $F(x)=
      \begin{cases}
        f\left(x^{3}\sin\frac{1}{x}\right), & x\neq0, \\
        f(0),                               & x=0.
      \end{cases}$

      当$x\neq0$时,$F'(x)=f'\left(x^{3}\sin\frac{1}{x}\right)\left(3x^{2}\sin\frac{1}{x}
      -x\cos\frac{1}{x}\right)$,

      当$x=0$时,
      \[
        \begin{aligned}
          g'(0)=  & \lim_{x\to0}\frac{x^3\sin\frac{1}{x}-0}{x}=0, \\
          F'(0) = & f'\left(g(0)\right)\cdot g'(0)=f'(0)\cdot0=0.
        \end{aligned}
      \]

    \item[15.] \textbf{Solution}. 设时刻$t$容器中水的体积为$V\mathrm{m}^{3}$,水的高度为为$h
      \mathrm{m}$,水面半径为$r\mathrm{m}$,则
      \[
        \frac{r}{4}=\frac{h}{8},\quad\text{即}\,r=\frac{h}{2},\quad\text{因而}\,V
        =\frac{1}{12}\pi h^{3},
      \]

      于是$\frac{\mathrm{d}V}{\mathrm{d}t}=\frac{1}{4}\pi h^{2}\cdot\frac{\mathrm{d}h}{\mathrm{d}t}$.

      当$h=5\mathrm{m}$时,$\frac{\mathrm{d}V}{\mathrm{d}t}=4\mathrm{m}^{3}/\mathrm{min}$时,$\frac{\mathrm{d}h}{\mathrm{d}t}
      =\frac{16}{25\pi}\mathrm{m}/\mathrm{min}$.

      即水面升高的速率是$\frac{16}{25\pi}\mathrm{m}/\mathrm{min}$.
  \end{enumerate}

  \section{证明题(每小题 5 分,共 10 分)}
  \begin{enumerate}
    \item[16.] \textbf{Proof}.由$x_{n+1}-x_{n}<4-\frac{4}{x_{n}}-x_{n}=-\left(\sqrt{x_{n}}
      -\frac{2}{\sqrt{x_{n}}}\right)^{2}\leq0$得到$\{x_{n}\}$单调递减.

      (或由均值不等式$\sqrt{x_{n+1}\cdot\frac{4}{x_n}}\leq\frac{x_{n+1}+\frac{4}{x_n}}{2}$及$x
      _{n+1}+\frac{4}{x_{n}}<4$得

      $\sqrt{x_{n+1}\cdot\frac{4}{x_n}}<2$,即$\frac{x_{n+1}}{x_{n}}<1$,从而$\{x
      _{n}\}$单调递减.)

      又$x_{n}>0$,由单调有界准则可知$\lim_{n\to\infty}x_{n}$存在.

      设$\lim_{n\to\infty}x_{n}=A$,则$A>0$.由$x_{n+1}+\frac{4}{x_{n}}<4$得$\lim_{n\to\infty}
      \left(x_{n+1}+\frac{4}{x_{n}}\right)\leq 4$,即$A+\frac{4}{A}\leq4$,解得$A
      =2$.

    \item[17.] \textbf{Proof}. 设$f(x)$在$x_{0}\in(0,b)$处取得最大值,则$f'(x_{0}
      )=0$.

      对函数$f'(x)$在$[0,x_{0}],[x_{0},b]$上分别应用Lagrange中值定理,得
      \[
        \begin{aligned}
          f'(x_{0})-f'(0)=f''(\eta)x_{0},\quad    & \exists\,\eta\in(0,x_{0}) \\
          f'(b)-f'(x_{0})=f''(\xi)(b-x_{0}),\quad & \exists\,\xi\in(x_{0},b)
        \end{aligned}
      \]

      $|f'(0)|+|f'(b)|=|f''(\eta)|x_{0}+|f''(\xi)|(b-x_{0})\leq Mx_{0}+M(b-x_{0})
      =Mb$.
  \end{enumerate}
\end{document}
