\documentclass[../Main.tex]{subfiles}
\begin{document}
\chapter{2023-2024学年微积分(一)(上)期中考试}


\section{基本计算题(每小题 6 分,共 60 分)}
\begin{enumerate}
    \item 求极限$l=\lim_{n\to\infty}(-1)^n\sin\left(\sqrt{n^2+1}\pi\right)$.
    \vspace{11em}
    \item 求极限$l=\lim_{x\to0}\frac{\mathrm{e}^{\tan x}-\mathrm{e}^{x}}{x-\sin x}.$
    \vspace{11em}    
    \item 当$x\to 0$时,设$u=\sqrt{1+\tan x}-\sqrt{1+\sin x}$的主部为$cx^k$,试确定常数$c,k$的值.
    \vspace{11em}
    \item 求极限$l=\lim_{x\to0}\left(\frac{\arcsin x}{x}\right)^{\frac{1}{x^2}}$.
    \vspace{11em}
    \item 设函数$f(x)=\frac{x-x^2}{\sin \pi x}$,指出$f(x)$的间断点并判断其类型.
    \vspace{9em}
    \item 设$y=\sqrt{x\sin x\sqrt{\mathrm{e}^x-1}}(0<x<\pi)$,求导数$y'$.
    \vspace{10em}
    \item 设$f(x)=\begin{cases}
        a+x+\sqrt{1-x},&x<0\\
        1+b\arctan x,&x\geq0
    \end{cases}$,求常数$a,b$,使$f(x)$在$x=0$处可导.
    \vspace{10em}
    \item 设函数$y=y(x)$由方程$y=2\mathrm{e}^y\sin x-7x$确定的可微函数,求$\left.\mathrm{d}y\right|_{x=0}$.
    \vspace{10em}
    \item 设$y=\frac{1}{2-3x-2x^2}$,求$y^{(n)}(0)$.
    \vspace{8em}
    \item 设$y=\sin^4 x-\cos ^4 x$,求$y'$及$y''$.
    \vspace{8em}
\end{enumerate}

\section{综合题(每小题 6 分,共 30 分)}
\begin{enumerate}
    \item[11.] 设函数$y=y(x)$由参数方程$\begin{cases}
        x=t^2+1,\\
        y=4t-t^2
    \end{cases},t>0$确定,过点$(-1,0)$作$y=y(x)$的切线,求切点$(x_0,y_0)$,并写出该切线的方程.
    \vspace{10em}
    \item[12.] 设$y=g(x)$是$y=f(x)$的反函数,$f(x)=x+(1+x)^x$,求$y=g(2+x^2)$在$x=1$处的导数.
    \vspace{10em}
    \item[13.] 设$b_i\geq0(i=1,2,\cdots,n,\cdots)$,$x_n=\frac{b_1}{1+b_1}+\frac{b_2}{(1+b_1)(1+b_2)}+\cdots+\frac{b_n}{(1+b_1)(1+b_2)\cdots(1+b_n)}$,研究数列$\{x_n\}$的极限的存在性.
    \vspace{8em}
    \item[14.] 设函数$f(x)=\begin{cases}
        \frac{g(x)-\mathrm{e}^{-x}}{x},&x\neq0,\\
        0,&x=0
    \end{cases}$,其中$g(x)$在$(-\infty,+\infty)$内具有一阶连续的导数,且$g''(0)$存在,$g(0)=1$,$g'(0)=-1$,求$f'(x)$,并讨论$f'(x)$在$x=0$处的连续性.
    \vspace{8em}
    \item[15.] 一观察者站在地面上用望远镜观察一架飞机,该飞机的高度为$20\mathrm{km}$,正以$16\mathrm{km/min}$的速度向观察者水平地飞过来,试问当飞机离观察者的水平距离为$40\mathrm{km}$时,望远镜视角改变的速率是多少?
    \vspace{8em}
\end{enumerate}


\section{证明题(每小题 5 分,共 10 分)}
\begin{enumerate}
    \item[16.] 设函数$f(x)$在$(a,b)$内连续,且$x_i\in(a,b),i=1,2,\cdots,n$,证明至少存在一点$\xi\in(a,b)$,使
    \[
    f(\xi)=\frac{1}{n}\left(f(x_1)+f(x_2)+\cdots+f(x_n)\right).
    \]
    \vspace{10em}
    \item[17.] 设函数$f(x)$在$[0,1]$上连续,在$(0,1)$内可导,并且$f(0)=0,f(1)=\frac{1}{2}$,证明存在$\xi,\eta\in(0,1)$,且$\xi\neq\eta$,使得$f'(\xi)+f'(\eta)=\xi+\eta$.
    \vspace{8em}
\end{enumerate}


\chapter{2023-2024学年微积分(B)(上)期中考试参考答案}
\section{基本计算题(每小题 6 分,共 60 分)}
\begin{enumerate}
    \item \textbf{Solution}. 
    \[
    \begin{aligned}
        l=&\lim_{n\infty}(-1)^n n\sin\left((\sqrt{n^2+1}-n)\pi+n\pi\right)\\
        =&\lim_{n\to\infty}n\sin\left((\sqrt{n^2+1}-n)\pi\right)=\lim_{n\to\infty}n\sin\left(\frac{\pi}{\sqrt{n^2+1}+n}\right)\\
        =&\lim_{n\to\infty}\frac{n}{\sqrt{n^2+1}+n}\pi=\frac{\pi}{2}.
    \end{aligned}
    \]

    \item \textbf{Solution}.
    \[
        \begin{aligned}
        l=&\lim_{x\to 0}\frac{\mathrm{e}^{x}(\mathrm{e}^{\tan x -x}-1)}{x-\sin x}=\lim_{x\to0}\frac{\tan x-x}{x-\sin x}\\
        =&\lim_{x\to0}\frac{\sec ^2 x-1}{1-\cos x}=\lim_{x\to0}\frac{\tan^2 x}{\frac{1}{2}x^2}=2.
        \end{aligned}
    \]

    或
        \[
        \begin{aligned}
        l=&\lim_{x\to 0}\mathrm{e}^\xi\frac{\tan x -x}{x-\sin x} \qquad (\xi\,\text{介于}\,x\,\text{与}\tan x\,\text{之间})\\
        =&\lim_{x\to 0}\frac{\tan x -x}{x-\sin x}\\
        =&\lim_{x\to0}\frac{\sec ^2 x-1}{1-\cos x}=\lim_{x\to0}\frac{\tan^2 x}{\frac{1}{2}x^2}=2.
        \end{aligned}
    \]

    \item \textbf{Solution}.
    
    \textbf{法一.}
    \[
    \begin{aligned}
        u=&\frac{1}{\sqrt{1+\tan x}+\sqrt{1+\sin x}}(\tan x-\sin x)\\
        \sim&\frac{1}{2}(\tan x-\sin x)=\frac{1}{2}\tan x(1-\cos x)\sim\frac{1}{4}x^3,
    \end{aligned}
    \]

    即$k=3,c=\frac{1}{4}$.

    \textbf{法二.}由题意得$\lim_{x\to0}\frac{u}{cx^k}=1$.
    而
    \[
    \begin{aligned}
        \lim_{x\to0}\frac{u}{cx^k}=&\frac{\tan x-\sin x}{cx^k(\sqrt{1+\tan x}+\sqrt{1+\sin x})}\\
        =&\lim_{x\to0}\frac{\tan x(1-\cos x)}{2cx^k}\\
        =&\lim_{x\to0}\frac{\frac{1}{2}x^3}{2cx^k}=\lim_{x\to0}\frac{x^{3-k}}{4c},
    \end{aligned}
    \]
    
    故$3-k=0,4c=1$,即$k=3,c=\frac{1}{4}$.

    \item \textbf{Solution}.
    \[
    \begin{aligned}
        l=&\lim_{x\to0}\mathrm{e}^{\frac{1}{x^2}\ln\left(\frac{\arcsin x}{x}\right)}=\mathrm{e}^{\lim_{x\to0}\frac{1}{x^2}\ln\left(\frac{\arcsin x}{x}\right)}\\\
        =&\mathrm{e}^{\lim_{x\to0}\frac{1}{x^2}\left(\frac{\arcsin x}{x}-1\right)}=\mathrm{e}^{\lim_{x\to0}\frac{\arcsin x-x}{x^3}}=\mathrm{e}^{\lim_{x\to0}\frac{\frac{1}{\sqrt{1-x^2}}-1}{3x^2}}=\mathrm{e}^{\lim_{x\to0}\frac{1-\sqrt{1-x^2}}{3x^2\sqrt{1-x^2}}}\\
        =&\mathrm{e}^{\lim_{x\to0}\frac{\frac{1}{2}x^2}{3x^2\sqrt{1-x^2}}}=\mathrm{e}^{\frac{1}{6}}.
    \end{aligned}
    \]

    \item \textbf{Solution}.
    
    当$\sin\pi x=0$即$x=k,k=0,\pm1,\pm2,\cdots$时,$f(x)$无定义,故$x=k,k=0,\pm1,\pm2\cdots$是$f(x)$的间断点.

    因
    \[
    \begin{aligned}
        \lim_{x\to0}f(x)=&\lim_{x\to0}\frac{x-x^2}{\sin\pi x}=\lim_{x\to0}\frac{\pi x}{\sin\pi x}\cdot\frac{1-x}{\pi}=\frac{1}{\pi};\\
        \lim_{x\to1}f(x)=&\lim_{x\to1}\frac{\pi(1-x)}{\sin\pi(1-x)}\cdot\frac{x}{\pi}=\frac{1}{\pi},
    \end{aligned}
    \]

    所以$x=0,1$是可去间断点;

    当$k=-1,\pm2,\pm3,\cdots$时,$\lim_{x\to k}f(x)=\infty$,所以$x=-1,\pm2,\pm3,\cdots$是无穷间断点.

    \item \textbf{Solution}.因$\ln y=\frac{1}{2}\left(\ln x+\ln\sin x+\frac{1}{2}\ln\left(\mathrm{e}^x-1\right)\right)$,
    
    所以$y'=\frac{1}{2}\sqrt{x\sin x\sqrt{\mathrm{e}^x-1}}\left(\frac{1}{x}+\cot x+\frac{\mathrm{e}^x}{2(\mathrm{e}^x-1)}\right)$.

    \item \textbf{Solution}.要$f(x)$在$x=0$处可导,必须$f(x)$在$x=0$处连续,即
    \[
    \lim_{x\to0^+}f(x)=\lim_{x\to0^-}f(x)=f(0).
    \]
    
    因$f(0)=1$,$\lim_{x\to0^+}f(x)=\lim_{x\to0^+}(1+b\arctan x)=1=f(0)$,

    $\lim_{x\to0^-}f(x)=\lim_{x\to0^-}(a+x+\sqrt{1-x})=a+1$,所以$a+1=1$,即$a=0$.

    又$f'_-(0)=\lim_{x\to0}\frac{(x+\sqrt{1-x})-1}{x}=\lim_{x\to0}\frac{\sqrt{1-x}(1-\sqrt{1-x})}{x}=\lim_{x\to0}\frac{\sqrt{1-x}\cdot\frac{1}{2}x}{x}=\frac{1}{2}$,

    $f'_+(0)=\lim_{x\to0}\frac{1+b\arctan x-1}{x}=\lim_{x\to0}\frac{b\arctan x}{x}=b$,

    要$f(x)$在$x=0$处可导,必须$f'_-(0)=f'_+(0)$,所以$b=\frac{1}{2}$.

    \item \textbf{Solution}.

    \textbf{法一.} 当$x=0$时$y=0$.

    方程两边求导
    \[
    y'=2\mathrm{e}^yy'\sin x+2\mathrm{e}^y\cos x-7,
    \]
    
    即$y'=\frac{2\mathrm{e}^y\cos x-7}{1-2\mathrm{e}^y\sin x}$,

    所以$\left.y'\right|_{x=0}=-5$,$\left.\mathrm{d}y\right|_{x=0}=\left.y'\right|_{x=0}\mathrm{d}x=-5\mathrm{d}x$.

    \textbf{法二.} 当$x=0$时$y=0$.

    方程两边求微分,得
    \[
    \mathrm{d}y=2\mathrm{e}^y\sin x\mathrm{d}y+2\mathrm{e}^y\cos x\mathrm{d}x-7\mathrm{d}x,
    \]

    $\mathrm{d}y=\frac{2\mathrm{e}^y\cos x-7}{1-2\mathrm{e}^y\sin x}\mathrm{d}x$,$\left.\mathrm{d}y\right|_{x=0}=\left.\frac{2\mathrm{e}^y\cos x-7}{1-2\mathrm{e}^y\sin x}\right|_{x=0}\mathrm{d}x=-5\mathrm{d}x$.

    \item \textbf{Solution}.$y=\frac{1}{(1-2x)(x+2)}=\frac{1}{5}\cdot\frac{1}{x+2}-\frac{2}{5}\cdot\frac{1}{2x-1}$.
    
    由公式$\left(\frac{1}{(1+x)}\right)^{(n)}=\frac{(-1)^n\,n!}{(1+x)^{n+1}}$,得
    \[
    \begin{aligned}
        y^{(n)}(x)=&\frac{1}{5}\left(\frac{1}{x+2}\right)^{(n)}-\frac{2}{5}\left(\frac{1}{2x-1}\right)^{(n)}=\frac{1}{5}\frac{(-1)^n\cdot n!}{(x+2)^{n+1}}-\frac{2}{5}\frac{(-1)^n\cdot2^n\cdot n!}{(2x-1)^{n+1}}\\
        =&\frac{n!}{5}\left(\frac{(-1)^n}{(x+2)^{n+1}}-\frac{(-1)^n2^{n+1}}{(2x-1)^{n+1}}\right),
    \end{aligned}
    \]

    所以$y^{(n)}(0)=\frac{n!}{5}\left(\frac{(-1)^n}{2^{n+1}}+2^{n+1}\right)$.

    \item \textbf{Solution}.$y=\sin^4 x-\cos^4 x=(\sin ^2 x+\cos^2 x)(\sin^2 x-\cos^2 x)=-\cos 2x$,
    
    所以
    \[
    \begin{aligned}
        y'=&2\sin 2x,\\
        y''=&4\cos 2x.
    \end{aligned}
    \]
\end{enumerate}

\section{综合题(每小题 6 分,共 30 分)}
\begin{enumerate}
    \item[11.] \textbf{Solution}. 设切点$(x_0,y_0)$对应的参数为$t=t_0$,则$\left.\frac{\mathrm{d}y}{\mathrm{d}x}\right|_{t=t_0}=\left.\frac{4-2t}{2t}\right|_{t=t_0}=\frac{2-t_0}{t_0}$,
    
    曲线在$(x_0,y_0)$的切线方程为$y-y_0=\frac{2-t_0}{t_0}(x-x_0)$,

    即$y-(4t_0-t_0^2)=\frac{2-t_0}{t_0}(x-t_0^2-1)$化简得$y=\frac{2-t_0}{t_0}x+2t_0-\frac{2-t_0}{t_0}$,

    切线过$(-1,0)$,故代入方程得$t_0^2-t_0-2=0$,解得:$t_0=-2$(不合题意),$t_0=1$,

    由$t_0=1$得$(x_0,y_0)=(2,3)$,故所求切线方程为$y=x+1$.

    \item[12.] \textbf{Solution}. 由题意得$f'(x)=1+(1+x)^x\left(\ln(1+x)+\frac{x}{1+x}\right)$,
    
    所以$f'(1)=2(\ln2+1)$.
    
    $f(1)=3$,$g'(3)=\frac{1}{f'(1)}=\frac{1}{2(\ln2+1)}$,

    $y'=g'(2+x^2)\cdot 2x=2xg'(2+x^2)$,$\left.y'\right|_{x=1}=g'(3)\cdot2=\frac{1}{\ln2+1}$.

    \item[13.] \textbf{Solution}. 
    
    \[
    \begin{aligned}
       x_{n+1}=&\frac{b_1}{1+b_1}+\frac{b_2}{(1+b_1)(1+b_2)}+\cdots+\frac{b_n}{(1+b_1)(1+b_2)\cdots(1+b_n)}+\frac{b_{n+1}}{(1+b_1)(1+b_2)\cdots(1+b_{n+1})}\\
       =&x_n+\frac{b_{n+1}}{(1+b_1)(1+b_2)\cdots(1+b_{n+1})}\geq x_n,
    \end{aligned}
    \] 

    故数列$\{x_n\}$单调递增.

    因$\frac{b_k}{(1+b_1)(1+b_2)\cdots(1+b_k)}=\frac{1}{(1+b_1)(1+b_2)\cdots(1+b_{k-1})}-\frac{1}{(1+b_1)(1+b_2)\cdots(1+b_k)}$,

    所以$x_n=1-\frac{1}{1+b_1}+\left(\frac{1}{1+b_1}-\frac{1}{(1+b_1)(1+b_2)}\right)+\cdots+\left(\frac{1}{(1+b_1)(1+b_2)\cdots(1+b_{n-1})}-\frac{1}{(1+b_1)(1+b_2)\cdots(1+b_n)}\right)$,

    故$x_n=1-\frac{1}{(1+b_1)(1+b_2)\cdots(1+b_n)}<1$.

    从而数列$\{x_n\}$有上界,由单调有界准则知数列得极限存在.
    
    \item[14.] \textbf{Solution}. 
    
    当$x\neq0$时,$f'(x)=\frac{x(g'(x)+\mathrm{e}^{-x})-g(x)+\mathrm{e}^{-x}}{x^2}=\frac{xg'(x)+(x+1)\mathrm{e}^{-x}-g(x)}{x^2}$.

    当$x=0$时,
    \[
    \begin{aligned}
    f'(0)=&\lim_{x\to0}\frac{\frac{g(x)-\mathrm{e}^{-x}}{x}-0}{x}=\lim_{x\to0}\frac{g(x)-\mathrm{e}^{-x}}{x^2}=\lim_{x\to0}\frac{g'(x)+\mathrm{e}^{-x}}{2x}\\
    =&\lim_{x\to0}\frac{g'(x)+1+\mathrm{e}^{-x}-1}{2x}=\lim_{x\to0}\frac{g'(x)-g'(0)+\mathrm{e}^{-x}-1}{2x}=\frac{1}{2}(g''(0)-1),
    \end{aligned}
    \]

    因
    \[
    \begin{aligned}
        \lim_{x\to 0}f'(x)=&\lim_{x\to0}\frac{x(g'(x)+\mathrm{e}^{-x})}{x^2}=\lim_{x\to0}\left(\frac{g'(x)+\mathrm{e}^{-x}}{x}-\frac{g(x)-\mathrm{e}^{-x}}{x^2}\right)\\
        =&\lim_{x\to0}\frac{g'(x)+\mathrm{e}^{-x}}{x}=-\lim_{x\to0}\frac{g(x)-\mathrm{e}^{-x}}{x^2}=g''(0)-1-\frac{1}{2}(g''(0)-1)=\frac{1}{2}(g''(0)-1)=f'(0),
    \end{aligned}
    \]

    故$f'(x)$在$x=0$处连续.

    \item[15.] \textbf{Solution}. 设时刻$t$飞机离观察者的水平距离为$x\mathrm{km}$,望远镜视角为$\theta$弧度,则有
    \[
    \frac{\mathrm{d}x}{\mathrm{d}t}=-16\mathrm{km/min},\quad x=20\tan\theta,
    \]
    
    故$\theta=\arctan\frac{x}{20}$,于是$\frac{\mathrm{d}\theta}{\mathrm{d}t}=\frac{1}{20+\frac{x^2}{20}}\frac{\mathrm{d}x}{\mathrm{d}t}$.
   
    当$x=40\mathrm{km}$,$\frac{\mathrm{d}x}{\mathrm{d}t}=-16(\mathrm{km/min})$时,$\frac{\mathrm{d}\theta}{\mathrm{d}t}=\frac{1}{20+\frac{40^2}{20}}(-16)=-\frac{4}{25}\mathrm{rad/min}$.

    即望远镜视角改变的速率是$-\frac{4}{25}\mathrm{rad/min}$.

\end{enumerate}

\section{证明题(每小题 5 分,共 10 分)}
\begin{enumerate}
    \item[16.] \textbf{Proof}.不妨设$a<x_1<x_2<\cdots<x_n<b$,显然$f(x)$在$[x_1,x_n]$上连续.
    
    故$f(x)$在$[x_1,x_n]$上必取得最大值$M$和最小值$m$,即$m\leq f(x)\leq M,x\in[x_1,x_n]$,从而有$m\leq f(x_i)\leq M,i=1,2,\cdots,n$,故有
    \[
    m\leq\frac{1}{n}\left(f(x_1)+f(x_2)+\cdots+f(x_n)\right)\leq M.
    \]

    由介值定理知至少存在一点$\xi\in[x_1,x_n]\subset(a,b)$使得
    \[
    f(\xi)=\frac{1}{n}\left(f(x_1)+f(x_2)+\cdots+f(x_n)\right).
    \]
    
    使得$f(\xi)=0$即$\alpha\xi=\tan\xi$,故方程$\alpha x=\tan x$在$\left(0,\frac{\pi}{2}\right)$内至少有一个实根.

    \item[17.] \textbf{Proof}. 
    
    \textbf{法一.} 令$F(x)=f(x)-\frac{1}{2}x^2$,

    则$F(x)$在$[0,1]$上连续,在$(0,1)$内可导,且$F(0)=f(0)=0$,$F(1)=f(1)-\frac{1}{2}=0$,

    函数$F(x)$在$\left[0,\frac{1}{2}\right],\left[\frac{1}{2},1\right]$上分别应用Lagrange中值定理得,$\exists\,\eta\in\left(0,\frac{1}{2}\right),\xi\in\left(\frac{1}{2},1\right)$,使得
    \[
    \begin{gathered}
    F\left(\frac{1}{2}\right)-F(0)=F'(\eta)\frac{1}{2},\qquad\text{即}\,2F\left(\frac{1}{2}\right)=f'(\eta)-\eta,\\
    F(1)-F\left(\frac{1}{2}\right)=F'(\xi)\frac{1}{2},\qquad\text{即}\,-2F\left(\frac{1}{2}\right)=f'(\xi)-\xi, 
    \end{gathered}
    \]

    两式相加得$f'(\eta)-\eta+f'(\xi)-\xi=0$,即$f'(\xi)+f'(\eta)=\xi+\eta$.

    \textbf{法二.} 令$G(x)=f(x)-f(1-x)$,则
    \[
    G'(x)=f'(x)+f'(1-x),\quad G(0)=f(0)-f(1)=-\frac{1}{2},\quad G\left(\frac{1}{2}\right)=0.
    \] 

    应用Lagrange中值定理得,$\exists\,\xi\in\left(0,\frac{1}{2}\right)$,使
    \[
    G\left(\frac{1}{2}\right)-G(0)=G'(\xi)\left(\frac{1}{2}-0\right)\quad\text{即}\quad f'(\xi)+f'(1-\xi)=1.
    \]

    令$\eta=1-\xi\in\left(\frac{1}{2},1\right)$,即得$f'(\xi)+f'(\eta)=\xi+\eta$.
\end{enumerate}
\end{document}
