\documentclass[../Main.tex]{subfiles}
\begin{document}
  \chapter{2024-2025学年微积分(B)(上)期中考试}

  \section{基本计算题(每小题 6 分,共 60 分)}
  \begin{enumerate}
    \item 设$x_{n}=\frac{1}{n+1}+\frac{1}{n+\sqrt[3]{2}}+\cdots+\frac{1}{n+\sqrt[3]{n}}$,求$\lim
      _{n\to\infty}x_{n}$.
      \vspace{11em}

    \item 求极限$l=\lim_{x\to0}\frac{\mathrm{e}^{x}-\mathrm{e}^{x\cos x}}{x\ln(1+x^{2})}
      .$
      \vspace{11em}

    \item 当$x\to0$时,求无穷小量$u=\left(\frac{2+\cos x}{3}\right)^{x^2}-1$的主部和阶数.
      \vspace{11em}

    \item 求极限$l=\lim_{x\to0}\frac{\sqrt{1-x^2}-\cos 3x}{\mathrm{e}^{x}-1-x}$.
      \vspace{8em}

    \item 设函数$f(x)=\frac{\sqrt{1+\sin x+\sin^{2}x}-(\alpha+\beta\sin x)}{\sin^{2}x}$,且点$x
      =0$是$f(x)$的可去间断点,试求常数$\alpha,\beta$的值.
      \vspace{9em}

    \item 设$y=\sqrt[3]{\frac{x(x^{2}+1)}{(x^{2}-1)^{2}}}$,求导数$y'$.
      \vspace{10em}

    \item 设$y=f(\sin^{2}x)$,求$y'$及$y''$.
      \vspace{10em}

    \item 设$f(x)=
      \begin{cases}
        (1+x^{2})^{\frac{1}{\sin x}}, & x\neq0, \\
        1,                            & x=0
      \end{cases}$,求$f'(x)$.
      \vspace{10em}

    \item 设函数$y=y(x)$是由方程$\cos(xy)+\ln y-x-1=0$确定的可微函数,求$\left.\mathrm{d}
      y\right|_{x=0}$.
      \vspace{10em}

    \item 设$f(x)=x^{2}\ln(1+x)$,求$f^{(6)}(0)$.
      \vspace{10em}
  \end{enumerate}

  \section{综合题(每小题 6 分,共 30 分)}
  \begin{enumerate}
    \item[11.] 讨论函数$f(x)=\lim_{n\to\infty}\frac{x^{2}\mathrm{e}^{n(x-1)}+ax+b}{\mathrm{e}^{n(x-1)}+1}$的连续性,其中$a
      ,b$为常数.
      \vspace{13em}

    \item[12.] 设函数$y=f(x)$由参数方程$\begin{cases}
        x=\arctan\sqrt{t}, \\
        y=y(t)
      \end{cases}(t>0)$确定,其中$y=y(t)$由方程$\mathrm{e}^{y}+\mathrm{e}^{t}=2\mathrm{e}$确定,求曲线$y
      =f(x)$在点$\left(\frac{\pi}{4},f\left(\frac{\pi}{4}\right)\right)$处的切线方程.
      \vspace{13em}

    \item[13.] 设数列$\{x_{n}\}$满足$x_{1}=1$,$x_{n}=x_{n-1}+\frac{1}{n^{2}}(n\geq
      2)$,研究极限$\lim_{n\to\infty}x_{n}$的存在性.
      \vspace{8em}

    \item[14.] 在距火箭发射塔4000米处安装一台摄像机,为使摄像机的镜头始终对准火箭,摄像机的仰角$\theta$应随着火箭的上升不断增加.假设火箭发射后垂直上升到距离地面3000米处,其速度为$6
      00$米$/\mathrm{s}$,试求在此时刻摄像机仰角$\theta$的变化率.
      \vspace{8em}

    \item[15.] 设可微函数$y=f(x)$由方程$x^{3}+y^{3}+xy-1=0$确定,求$\lim_{x\to0}\frac{f(\cos
      x-1)-1}{\ln(1+x^{2})}$.
      \vspace{8em}
  \end{enumerate}

  \section{证明题(每小题 5 分,共 10 分)}
  \begin{enumerate}
    \item[16.] 设正整数$n>1$,证明:$x^{2n}+a_{1}x^{2n-1}+a_{2}x^{2n-2}+\cdots+a_{2n-1}
      x-1=0$至少有两个实根.
      \vspace{10em}

    \item[17.] 设函数$f(x)$在$[0,1]$上连续,在$(0,1)$内可导,并且$f(0)=f(1)=0,f\left
      (\frac{1}{2}\right)=1$,证明存在$\xi\in(0,1)$,使得$f'(\xi)=1$.
      \vspace{8em}
  \end{enumerate}

  \chapter{2024-2025学年微积分(B)(上)期中考试参考答案}
  \section{基本计算题(每小题 6 分,共 60 分)}
  \begin{enumerate}
    \item \textbf{Solution}. 注意到$n+1\leq n+\sqrt[3]{k}\leq n+\sqrt[3]{n}$,所以
      \[
        \frac{n}{n+\sqrt[3]{n}}\leq x_{n}\leq \frac{n}{n+1}.
      \]

      且$\lim_{n\to\infty}\frac{n}{n+\sqrt[3]{n}}=\lim_{n\to\infty}\frac{n}{n+1}=
      1$,由夹逼定理得$\lim_{n\to\infty}x_{n}=1$.

    \item \textbf{Solution}.
      \[
        \begin{aligned}
          l=- & \lim_{x\to 0}\mathrm{e}^{x}\frac{\mathrm{e}^{x\cos x-x}-1}{x^3}=-\lim_{x\to0}\frac{x\cos x-x}{x^3} \\
          =   & \lim_{x\to0}\frac{1-\cos x}{x^2}=\lim_{x\to0}\frac{\frac{1}{2}x^2}{x^2}=\frac{1}{2}.
        \end{aligned}
      \]

    \item \textbf{Solution}.
      \[
        \begin{aligned}
          u=\left(\frac{2+\cos x}{3}\right)^{x^2}-1= & \exp\left({x^2\ln\frac{2+\cos x}{3}}\right)-1       \\
          \sim                                       & x^{2}\ln\frac{2+\cos x}{3}                          \\
          \sim                                       & x^{2}\ln\left(1-\frac{x^2}{6}\right)                \\
          \sim                                       & x^{2}\left(-\frac{x^2}{6}\right)=-\frac{1}{6}x^{4}.
        \end{aligned}
      \]

      所以$u$的主部为$-\frac{1}{6}x^{4}$,阶数为4.

    \item \textbf{Solution}.
      \[
        \begin{aligned}
          l= & \lim_{x\to0}\frac{(\sqrt{1-x^2}-1)+(1-\cos 3x)}{1+x+\frac{1}{2}x^2+o(x^2)-1-x}  \\
          =  & \lim_{x\to0}\frac{-\frac{1}{2}x^2+\frac{9}{2}x^2+o(x^2)}{\frac{1}{2}x^2+o(x^2)} \\
          =  & 8.
        \end{aligned}
      \]

    \item \textbf{Solution}.
      \[
        \begin{aligned}
          \lim_{x\to0}f(x)= & \lim_{x\to0}\frac{\sqrt{1+\sin x+\sin^2 x}-(\alpha+\beta\sin x)}{\sin^2 x}                                                                  \\
          =                 & \lim_{x\to0}\frac{1+\frac{1}{2}\sin x+\frac{1}{2}\sin^2 x-\frac{1}{8}(\sin x+\sin^2 x)^2+o(\sin x+\sin^2 x)^2-\alpha-\beta\sin x}{\sin^2 x} \\
          =                 & \lim_{x\to0}\frac{1+\frac{1}{2}\sin x+\frac{1}{2}\sin^2 x-\frac{1}{8}\sin^2 x+o(\sin^2 x)-\alpha-\beta\sin x}{\sin^2 x}                     \\
          =                 & \lim_{x\to0}\frac{(1-\alpha)+(\frac{1}{2}-\beta)\sin x+\frac{3}{8}\sin^2 x+o(\sin^2 x)}{\sin^2 x}.
        \end{aligned}
      \]

      因为点$x=0$是$f(x)$的可去间断点,故$\lim_{x\to0}f(x)$存在,所以$1-\alpha=0$,$\frac{1}{2}
      -\beta=0$,解得$\alpha=1,\beta=\frac{1}{2}$.

    \item \textbf{Solution}. $\ln y=\frac{1}{3}\left[\ln x+\ln(x^{2}+1)-2\ln(x^{2}
      -1)\right]$,

      所以$\frac{1}{y}\cdot y'=\frac{1}{3}\left(\frac{1}{x}+\frac{2x}{x^{2}+1}-\frac{4x}{x^{2}-1}
      \right)$,即
      \[
        y'=\sqrt[3]{\frac{x(x^{2}+1)}{(x^{2}-1)^{2}}}\cdot\frac{1}{3}\left(\frac{1}{x}
        +\frac{2x}{x^{2}+1}-\frac{4x}{x^{2}-1}\right).
      \]

    \item \textbf{Solution}.

      $y'=f'(\sin^{2}x)\cdot2\sin x\cos x=f'(\sin^{2}x)\cdot\sin 2x$,

      $y''=f''(\sin^{2}x)\cdot\sin^{2}2x+2\cos 2x\cdot f'(\sin^{2}x)$.

    \item \textbf{Solution}.$f(x)$的定义域为$x\neq k\pi,k=\pm1, \pm2,\cdots$.

      当$x\neq0$时,$\ln f(x)=\frac{\ln(1+x^{2})}{\sin x}$,所以
      \[
        \begin{aligned}
          f'(x)= & f(x)\cdot\left(\frac{\frac{2x}{1+x^2}\sin x-\ln(1+x^2)\cos x}{\sin^2 x}\right)                            \\
          =      & (1+x^{2})^{\frac{1}{\sin x}}\cdot\left(\frac{\frac{2x}{1+x^2}\sin x-\ln(1+x^2)\cos x}{\sin^2 x}\right)    \\
          =      & (1+x^{2})^{\frac{1}{\sin x}}\cdot\left(\frac{2x}{(1+x^2)\sin x}-\frac{\ln(1+x^2)\cos x}{\sin^2 x}\right).
        \end{aligned}
      \]

      当$x=0$时,
      \[
        \begin{aligned}
          f'(0)= & \lim_{x\to0}\frac{(1+x^2)^{\frac{1}{\sin x}}-1}{x}=\lim_{x\to0}\frac{\mathrm{e}^{\frac{\ln(1+x^2)}{\sin x}}-1}{x} \\
          =      & \lim_{x\to 0}\frac{\ln(1+x^2)}{x\sin x}=1.
        \end{aligned}
      \]

      综上所述,$f'(x)=
      \begin{cases}
        (1+x^{2})^{\frac{1}{\sin x}}\cdot\left(\frac{2x}{(1+x^2)\sin x}-\frac{\ln(1+x^2)\cos x}{\sin^2 x}\right), & x\neq0, \\
        1,                                                                                                        & x=0.
      \end{cases}$

    \item \textbf{Solution}. 对方程$\cos(xy)+\ln y - x - 1=0$两边求微分得
      \[
        -\sin(xy)(y\mathrm{d}x+x\mathrm{d}y)+\frac{1}{y}\mathrm{d}y-\mathrm{d}x=0
        .
      \]

      当$x=0$时,$y=1$,代入上式得$\left.\mathrm{d}y\right|_{x=0}=\mathrm{d}x$.

    \item \textbf{Solution}. 利用$\ln(1+x)=\sum_{n=1}^{\infty}(-1)^{n-1}\frac{x^{n}}{n}$,得
      \[
        f(x)=x^{3}-\frac{1}{2}x^{4}+\frac{1}{3}x^{5}-\frac{1}{4}x^{6}+\cdots=\sum
        _{n=1}^{\infty}(-1)^{n-1}\frac{x^{n+2}}{n}.
      \]

      由Taylor定理,将上式与$f(x)=\sum_{n=0}^{\infty}\frac{f^{(n)}(0)}{n!}x^{n}$对照,得$f
      ^{(6)}(0)=-\frac{1}{4}\cdot 6!=-180.$
  \end{enumerate}

  \section{综合题(每小题 6 分,共 30 分)}
  \begin{enumerate}
    \item[11.] \textbf{Solution}.

      当$x>1$时,$\mathrm{e}^{n(x-1)}\to+\infty$,所以
      \[
        \begin{aligned}
          f(x)= & \lim_{n\to\infty}\frac{x^2\mathrm{e}^{n(x-1)}+ax+b}{\mathrm{e}^{n(x-1)}+1}                    \\
          =     & \lim_{n\to\infty}\frac{x^2+\frac{ax+b}{\mathrm{e}^{n(x-1)}}}{1+\frac{1}{\mathrm{e}^{n(x-1)}}} \\
          =     & x^{2}.
        \end{aligned}
      \]

      当$x=1$时,$\mathrm{e}^{n(x-1)}=1$,所以$f(1)=\left.\lim_{n\to\infty}\frac{x^{2}+ax+b}{2}
      \right|_{x=1}=\frac{1+a+b}{2}$.

      当$x<1$时,$\mathrm{e}^{n(x-1)}\to0$,所以$f(x)=\lim_{n\to\infty}\frac{ax+b}{1}
      =ax+b$.

      综上所述,$f(x)=
      \begin{cases}
        x^{2},           & x>1, \\
        \frac{1+a+b}{2}, & x=1, \\
        ax+b,            & x<1.
      \end{cases}$

      $f(x)$连续当且仅当$\begin{cases}
        1=\frac{1+a+b}{2}, \\
        a+b=1
      \end{cases}$,即$a+b=1$.

    \item[12.] \textbf{Solution}. 当$x=\frac{\pi}{4}$时,$t=1$,由方程$\mathrm{e}
      ^{y}+\mathrm{e}^{t}=2\mathrm{e}$得$y(1)=1$,所以$f\left(\frac{\pi}{4}\right
      )=1$.

      对方程$\mathrm{e}^{y}+\mathrm{e}^{t}=2\mathrm{e}$两边求微分得$\mathrm{e}^{y}
      \mathrm{d}y+\mathrm{e}^{t}\mathrm{d}t=0$,代入$t=1,y=1$得$\mathrm{d}y=-\mathrm{d}
      t$,所以$y'_{t}(1)=-1$.

      且$x'_{t}=\frac{1}{2\sqrt{t}(1+t)}$,所以$x'_{t}(1)=\frac{1}{4}$.故
      \[
        \left.\frac{\mathrm{d}y}{\mathrm{d}x}\right|_{x=\frac{\pi}{4}}=\frac{y'_{t}(1)}{x'_{t}(1)}
        =-4.
      \]

      所以切线方程为
      \[
        y-1=-4\left(x-\frac{\pi}{4}\right)\quad\text{或}\quad 4x+y-1-\pi=0.
      \]

    \item[13.] \textbf{Solution}. 由题可得

      \[
        \begin{aligned}
          x_{n}-x_{n-1}=   & \frac{1}{n^2},     \\
          x_{n-1}-x_{n-2}= & \frac{1}{(n-1)^2}, \\
          \cdots           &                    \\
          x_{2}- x_{1}=    & \frac{1}{2^2}.
        \end{aligned}
      \]

      将上述各式相加得$x_{n}=x_{1}+\frac{1}{2^{2}}+\frac{1}{3^{2}}+\cdots+\frac{1}{n^{2}}
      =\sum_{k=1}^{n}\frac{1}{k^{2}}$.

      显然$\{x_{n}\}$单调增加,且$x_{n}<1+\sum_{k=2}^{n}\frac{1}{k(k-1)}=2-\frac{1}{n}
      <2$,所以数列$\{ x_{n}\}$收敛,即$\lim_{n\to\infty}x_{n}$存在.

    \item[14.] \textbf{Solution}.

      设火箭上升的高度为$h$,则$\tan\theta=\frac{h}{4000}$,所以$\frac{\mathrm{d}h}{\mathrm{d}t}
      =4000\cdot\sec^{2}\theta\cdot\frac{\mathrm{d}\theta}{\mathrm{d}t}$.

      由题设可知此时刻$\frac{\mathrm{d}h}{\mathrm{d}t}=600$,$\sec\theta=\frac{\sqrt{4000^{2}+3000^{2}}}{4000}
      =\frac{5}{4}$,

      所以$\frac{\mathrm{d}\theta}{\mathrm{d}t}=\frac{\frac{\mathrm{d}h}{\mathrm{d}t}}{4000\cdot\sec^{2}\theta}
      =\frac{600}{4000\cdot\left(\frac{5}{4}\right)^{2}}=\frac{12}{125}$(弧度/秒).

    \item[15.] \textbf{Solution}. 当$x=0$时,由方程得$y=f(0)=1$.

      对方程$x^{3}+y^{3}+xy-1=0$两边求微分得$3x^{2}\mathrm{d}x+3y^{2}\mathrm{d}y+
      y\mathrm{d}x+x\mathrm{d}y=0$,所以当$x=0,y=1$时,$f'(0)=\left.\frac{\mathrm{d}y}{\mathrm{d}x}
      \right|_{x=0}=-\frac{1}{3}$.

      因此
      \[
        \begin{aligned}
          \lim_{x\to 0}\frac{f(\cos x-1)-1}{\ln(1+x^2)}= & \lim_{x\to0}\frac{f(\cos x-1)-f(0)}{\cos x -1 -0}\cdot\frac{\cos x -1}{\ln(1+x^2)} \\
          =                                              & f'(0)\cdot\lim_{x\to0}\frac{\cos x -1}{\ln(1+x^2)}                                 \\
          =                                              & -\frac{1}{3}\cdot\lim_{x\to0}\frac{-\frac{1}{2}x^2}{x^2}                           \\
          =                                              & \frac{1}{6}.
        \end{aligned}
      \]
  \end{enumerate}

  \section{证明题(每小题 5 分,共 10 分)}
  \begin{enumerate}
    \item[16.] \textbf{Proof}. 令$F(x)=x^{2n}+a_{1}x^{2n-1}+a_{2}x^{2n-2}+\cdots+
      a_{2n-1}x-1$,则$F(x)$在$\mathbf{R}$上连续.

      注意到$F(0)=-1,F(-\infty)=F(\infty)=+\infty$,由介值定理可知,存在$c_{1}\in
      (-\infty,0)$,$c_{2}\in(0,+\infty)$,

      使得$F(c_{1})=0$,$F(c_{2})=0$,即方程至少有两个实根.

    \item[17.] \textbf{Proof}. 令$F(x)=f(x)-x$,则$F(x)$在$[0,1]$上连续,在$(0,1)$内可导.

      注意到$F(0)=f(0)-0=0$,$F\left(\frac{1}{2}\right)=f\left(\frac{1}{2}\right)
      -\frac{1}{2}=\frac{1}{2}$,$F(1)=f(1)-1=-1$.

      任取$\eta\in\left(0,\frac{1}{2}\right)$,由介值定理可知存在$x_{1}\in\left(0
      ,\frac{1}{2}\right),x_{2}\in\left(\frac{1}{2},1\right)$,使得$F(x_{1})=F(x_{2}
      )=\eta$.

      所以由Rolle定理可知,存在$\xi\in(x_{1},x_{2})\subset(0,1)$,使得$F'(\xi)=0$,即$f
      '(\xi)=1$.
  \end{enumerate}
\end{document}
