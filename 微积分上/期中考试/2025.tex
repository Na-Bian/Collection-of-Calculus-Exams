\documentclass[../Main.tex]{subfiles}
\begin{document}
\chapter{2025-2026学年微积分(B)(上)期中考试}


\section{基本计算题(每小题 6 分,共 60 分)}
\begin{enumerate}
    \item 求极限$l=\lim_{n\to\infty}\left(\frac{1}{n^2+1}+\frac{2}{n^2+\sqrt{2}}+\cdots+\frac{n}{n^2+\sqrt{n}}\right).$
    \vspace{10em}
    \item 求极限$l=\lim_{x\to0}\frac{\mathrm{e}^x-\mathrm{e}^{\tan x}}{\ln(1+\sin^3 x)}.$
    \vspace{10em}    
    \item 设$y=\arctan \mathrm{e}^x-\ln\sqrt{\frac{\mathrm{e}^{2x}}{1+\mathrm{e}^{2x}}}$,求$\mathrm{d}y\big |_{x=1}$.
    \vspace{10em}
    \item 已知曲线的参数方程为
    $\begin{cases}
    x=\frac{3t}{1+t^3},\\
    y=\frac{3t^2}{1+t^3},
    \end{cases}$,求该曲线在对应$t=1$的点$P$处的切线方程.
    \vspace{10em}
    \item 求当$x\to0$时,无穷小量$u=\sqrt{1+x\arcsin x}-\sqrt{\cos x}$的主部和阶数.
    \vspace{10em}
    \item  讨论函数$f(x)=\lim_{n\to\infty}\frac{x+\mathrm{e}^{nx}}{1+\mathrm{e}^{nx}}$的连续性,若有间断点,指出其类型.
    \vspace{10em}
    \item 试确定常数$a,b$,使得$\lim_{x\to0}\frac{\sqrt{(1+x)^4+3}-(a+bx)}{x}=0$.
    \vspace{10em}
    \item 已知函数$y=f(\sin^2 x)$,其中$f(u)$二阶可导,求$y''$.
    \vspace{10em}
    \item 设$f(x)=\begin{cases}
        x^n \cos\frac{1}{x},&x\neq0,\\
        0,&x=0,
    \end{cases}$,要使$f'(x)$在$x=0$处连续,求正整数$n$的取值范围.
    \vspace{10em}
    \item 设$f(x)=\cos^4 x+\sin^4 x$,求$f^{(6)}(0)$.
    \vspace{10em}
\end{enumerate}

\section{综合题(每小题 6 分,共 30 分)}
\begin{enumerate}
    \item[11.] 设$y=y(x)$由方程$x=1+\frac{t}{2}+\frac{t^3}{6}$与$\mathrm{e}^y\sin t-y+1=0$确定,求$\frac{\mathrm{d}y}{\mathrm{d}x},\left.\frac{\mathrm{d}y}{\mathrm{d}x}\right|_{x=1}$.
    \vspace{13em}
    \item[12.] 设$f(x)$定义于$(-\infty,+\infty)$,且$f(x)\neq0$.若任给$x,y$总有$f(x+y)=f(x)f(y)$,且$f'(0)=\lambda$,讨论$f(x)$的可导性,并求$f(x)$.
    \vspace{13em}
    \item[13.] 设$f(x)$二阶可导,且$\lim_{x\to0}\frac{f(x)}{x}=0$,$f''(0)=2$,求$l=\lim_{x\to0}\left(1+\frac{f(x)}{x}\right)^{\frac{1}{x}}$.
    \vspace{8em}
    \item[14.] 设函数$f(u)$二阶可导且$f'(u)\neq0$,若$x=\varphi(y)$为$y=f(\ln x)$的反函数,求$\frac{\mathrm{d}^2 x}{\mathrm{d}y^2}$.
    \vspace{8em}
    \item[15.] 利用不等式$\frac{1}{n+1}<\ln\left(1+\frac{1}{n}\right)<\frac{1}{n}$,研究数列$x_n=1+\frac{1}{2}+\cdots+\frac{1}{n}-\ln n$的收敛性.
    \vspace{8em}
\end{enumerate}


\section{证明题(每小题 5 分,共 10 分)}
\begin{enumerate}
    \item[16.] 设常数$\alpha>1$,证明方程$\alpha x=\tan x$在$\left(0,\frac{\pi}{2}\right)$内至少有一个实根.
    \vspace{10em}
    \item[17.] 设函数$f(x)$在$[0,1]$上连续,在$(0,1)$内可导,并且$f(0)=f(1)=0,f(\theta)=1,\theta\in(0,1)$,证明:存在$\xi\in(0,1)$,使得$f'(\xi)-f(\xi)+\xi=1$.
    \vspace{8em}
\end{enumerate}


\chapter{2025-2026学年微积分(B)(上)期中考试参考答案}
\section{基本计算题(每小题 6 分,共 60 分)}
\begin{enumerate}
    \item \textbf{Solution}. 记和式$\frac{1}{n^2+1}+\frac{2}{n^2+\sqrt{2}}+\cdots+\frac{n}{n^2+\sqrt{n}}$为$A_n$,因为
    \[
        \frac{1}{2}\frac{n(n+1)}{n^2+\sqrt{n}}<A_n<\frac{1}{2}\frac{n(n+1)}{n^2+1},
    \]
    而$\lim_{n\to\infty}\frac{1}{2}\frac{n(n+1)}{n^2+\sqrt{n}}=\frac{1}{2}$,$\lim_{n\to\infty}\frac{1}{2}\frac{n(n+1)}{n^2+1}=\frac{1}{2}$,由夹逼准则得
    \[
    l=\lim_{n\to\infty}A_n=\frac{1}{2}.
    \]

    \item \textbf{Solution}.
    \[
        \begin{aligned}
        l=&-\lim_{x\to 0}\mathrm{e}^x\frac{\mathrm{e}^{\tan x-x}-1}{\sin^3 x}=-\lim_{x\to0}\frac{\tan x-x}{x^3}\\
        =&-\lim_{x\to0}\frac{\sec ^2 x-1}{3x^2}=-\lim_{x\to0}\frac{\tan^2 x}{3x^2}=-\frac{1}{3}.
        \end{aligned}
    \]

    \item \textbf{Solution}.
    
    $y=\arctan \mathrm{e}^x -x+\frac{1}{2}\ln(1+\mathrm{e}^{2x})$,

    $\mathrm{d}y=\frac{\mathrm{e}^x}{1+\mathrm{e}^{2x}}\mathrm{d}x+\frac{1}{2}\frac{1}{1+\mathrm{e}^{2x}}\mathrm{e}^{2x}\cdot2\mathrm{d}x=\frac{\mathrm{e}^x-1}{1+\mathrm{e}^{2x}}\mathrm{d}x$,

    $\mathrm{d}y\big |_{x=1}=\frac{\mathrm{e}-1}{1+\mathrm{e}^2}\mathrm{d}x$.

    \item \textbf{Solution}.对应$t=1$的点$P$为$\left(\frac{3}{2},\frac{3}{2}\right)$.
    
    $\begin{aligned}
        \frac{\mathrm{d}y}{\mathrm{d}x}=&\frac{y'_t}{x'_{t}}\\
        =&\frac{\frac{6t(1+t^3)-9t^4}{(1+t^3)^2}}{\frac{3(1+t^3)-9t^3}{(1+t^3)^2}}=\frac{2t-t^4}{1-2t^3}.
    \end{aligned}$

    曲线在点$P$的切线斜率为$k=\left.\frac{\mathrm{d}y}{\mathrm{d}x}\right|_{t=1}=-1$,故切线方程为
    \[
    y-\frac{3}{2}=-\left(x-\frac{3}{2}\right)\quad\text{或}\quad x+y=3.
    \]

    \item \textbf{Solution}.
    
    \textbf{法一.}设$u$的主部为$cx^k(c\neq0,k>0)$,则有
    \[
    \lim_{x\to0}\frac{\sqrt{1+x\arcsin x}-\sqrt{\cos x}}{cx^k}=1.
    \]

    而
    \[
    \begin{aligned}
        \lim_{x\to0}\frac{\sqrt{1+x\arcsin x}-\sqrt{\cos x}}{cx^k}=&\lim_{x\to0}\frac{1-\cos x+x\arcsin x}{cx^k(\sqrt{1+x\arcsin x}+\sqrt{\cos x})}\\
        =&\frac{1}{2c}\lim_{x\to0}\frac{1-\cos x+x\arcsin x}{x^k}=\frac{1}{2c}\lim_{x\to0}\frac{\frac{1}{2}x^2+x^2}{x^k}=\frac{3}{4c}\lim_{x\to0}{\frac{x^2}{x^k}},
    \end{aligned}
    \]

    要使$\frac{3}{4c}\lim_{x\to0}{\frac{x^2}{x^k}}=1$,必有$k=2,c=\frac{3}{4}$.

    综上所述,无穷小量$u=\sqrt{1+x\arcsin x}-\sqrt{\cos x}$的主部为$\frac{3}{4}x^2$,阶数为2.

    \textbf{法二.}当$x\to0$时,
    \[
        \begin{aligned}
            u=\sqrt{1+x\arcsin x}-\sqrt{\cos x}=&\frac{1-\cos x+x\arcsin x}{\sqrt{1+x\arcsin x}+\sqrt{\cos x}}\\
            \sim&\frac{1}{2}(1-\cos x+x\arcsin x)\sim\frac{1}{2}\left(\frac{1}{2}x^2+x^2\right)=\frac{3}{4}x^2.
        \end{aligned}
    \]

    综上所述,无穷小量$u=\sqrt{1+x\arcsin x}-\sqrt{\cos x}$的主部为$\frac{3}{4}x^2$,阶数为2.

    \item \textbf{Solution}.当$x<0$时,$\lim_{n\to\infty}\mathrm{e}^{nx}=0$,$f(x)=\lim_{n\to\infty}\frac{x+\mathrm{e}^{nx}}{1+\mathrm{e}^{nx}}=x$;
    
    当$x=0$时,$f(0)=\frac{1}{2}$;

    当$x>0$时,$\lim_{n\to\infty}\mathrm{e}^{-nx}=0$,$f(x)=\lim_{n\to\infty}\frac{x+\mathrm{e}^{nx}}{1+\mathrm{e}^{nx}}=\lim_{n\to\infty}\frac{x\mathrm{e}^{-nx}+1}{\mathrm{e}^{-nx}+1}=1.$

    因此$f(x)=\begin{cases}
        1,&x>0,\\
        \frac{1}{2},&x=0,\\
        x,&x<0
    \end{cases}$,在区间$(-\infty,0)\cup(0,+\infty)$连续.

    因$f(0^-)=0$,$f(0^+)=1$,故$x=0$是$f(x)$的跳跃间断点.

    \item \textbf{Solution}.由题意知$\sqrt{(1+x)^4+3}-(a+bx)=o(x)$,$(x\to0)$.
    
    首先,$\lim_{x\to0}\left(\sqrt{(1+x)^4+3}-(a+bx)\right)=0$,计算极限可得$a=2$.

    其次,$\lim_{x\to0}\frac{\sqrt{(1+x)^4+3}-(2+bx)}{x}=\lim_{x\to0}\left(\frac{\sqrt{(1+x)^4+3}-2}{x}-b\right)=0$,可得

    $b=\lim_{x\to0}\frac{\sqrt{(1+x)^4+3}-2}{x}=\lim_{x\to0}\frac{(1+x)^4-1}{x}\cdot\frac{1}{\sqrt{(1+x)^4+3}+2}=1.$

    \item \textbf{Solution}.$y'=f'(\sin^2 x)\cdot2\sin x\cos x=f'(\sin^2 x)\cdot\sin 2x$,
    
    $y''=f''(\sin^2 x)\cdot\sin^2 2x+2\cos 2x\cdot f'(\sin^2 x)$.
    
    \item \textbf{Solution}.要使$f'(x)$在$x=0$处连续,必有$\lim_{x\to0}f'(x)=f'(0).$
    
    要使$f'(0)=\lim_{x\to0}\frac{f(x)-f(0)}{x}=\lim_{x\to0}x^{n-1}\cos\frac{1}{x}$存在,必有$n>1$,此时$f'(0)=0$.

    $x\neq0$时,$f'(x)=nx^{n-1}\cos\frac{1}{x}+x^{n-2}\sin\frac{1}{x}$.当$n>1$时,
    \[
    \lim_{x\to0}f'(x)=\lim_{x\to0}\left(nx^{n-1}\cos\frac{1}{x}+x^{n-2}\sin\frac{1}{x}\right)=\lim_{x\to0}x^{n-2}\sin\frac{1}{x},
    \]

    欲使上式极限为0,必有正整数$n>2$.

    综上所述,当且仅当正整数$n>2$时,$f'(x)$在$x=0$处连续.

    \item \textbf{Solution}.因为$f(x)=1-2\cos^2 x\sin^2 x=1-\frac{1}{2}\sin^2 2x=\frac{3}{4}+\frac{1}{4}\cos 4x$,
    
    所以
    \[
    \begin{gathered}
        f^{(6)}(x)=4^5\cos\left(4x+\frac{6\pi}{2}\right)=-4^5\cos 4x,\\
        f^{(6)}(0)=-1024.
    \end{gathered}
    \]
\end{enumerate}

\section{综合题(每小题 6 分,共 30 分)}
\begin{enumerate}
    \item[11.] \textbf{Solution}. 当$t=0$时,$x=1,y=1$.$x'_{t}=\frac{1}{2}+\frac{t^2}{2}\neq0$.
    
    对方程$\mathrm{e}^y\sin t-y+1=0$两边关于$t$求导得
    \[
    \mathrm{e}^y\frac{\mathrm{d}y}{\mathrm{d}t}\sin t+\mathrm{e}^y\cos t-\frac{\mathrm{d}y}{\mathrm{d}t}=0,
    \]

    解得
    \[
    \begin{aligned}
        \frac{\mathrm{d}y}{\mathrm{d}t}=&\frac{\mathrm{e}^y\cos t}{1-\mathrm{e}^y\sin t},\\
        \frac{\mathrm{d}y}{\mathrm{d}x}=&\frac{y'_t}{x'_t}=\frac{2\mathrm{e}^y\cos t}{(1-\mathrm{e}^y\sin t)(1+t^2)},\\
        \left.\frac{\mathrm{d}y}{\mathrm{d}x}\right|_{x=1}=&\left.\frac{2\mathrm{e}^y\cos t}{(1-\mathrm{e}^y\sin t)(1+t^2)}\right|_{t=0}=2\mathrm{e}.
    \end{aligned}
    \]

    \item[12.] \textbf{Solution}. 由条件$f(x+y)=f(x)f(y)$得$f(0)=f^2(0)$,$f(x)=f^2\left(\frac{x}{2}\right)$.因$f(x)\neq0$,故$f(x)>0$,$f(0)=1$.
    \[
        \begin{aligned}
            \lim_{y\to0}\frac{f(x+y)-f(x)}{y}=&\lim_{y\to0}\frac{f(x)f(y)-f(x)}{y}\\
            =&f(x)\lim_{y\to0}\frac{f(y)-f(0)}{y}=f(x)f'(0)=\lambda f(x).
        \end{aligned}
    \]

    因此函数$f(x)$处处可导,且$f'(x)=\lambda f(x)$.

    因为$f'(x)-\lambda f(x)=0$,所以
    \[
    (\mathrm{e}^{-\lambda x}f(x))'=\mathrm{e}^{-\lambda x}f'(x)-\lambda \mathrm{e}^{-\lambda x}f(x)=0,
    \]

    从而$\mathrm{e}^{-\lambda x}f(x)\equiv k$,由$f(0)=1$得$k=1$,$f(x)=\mathrm{e}^{\lambda x}$.

    \item[13.] \textbf{Solution}. 由题设条件得$f(0)=0,f'(0)=0$.
    
    因为
    \[
    \begin{aligned}
       \lim_{x\to0}\ln\left(1+\frac{f(x)}{x}\right)^{\frac{1}{x}}=&\lim_{x\to0}\frac{1}{x}\ln\left(1+\frac{f(x)}{x}\right)=\lim_{x\to0}\frac{f(x)}{x^2}\\
       =&\lim_{x\to0}\frac{f'(x)}{2x}=\frac{1}{2}\lim_{x\to0}\frac{f'(x)-f'(0)}{x}=\frac{1}{2}f''(0)=1.
    \end{aligned}
    \] 

    所以
    \[
    l=\lim_{x\to0}\left(1+\frac{f(x)}{x}\right)^{\frac{1}{x}}=\mathrm{e}.
    \]
    
    \item[14.] \textbf{Solution}. 
    
    \textbf{法一.} 因$\frac{\mathrm{d}y}{\mathrm{d}x}=f'(\ln x)\cdot\frac{1}{x}$,所以$\frac{\mathrm{d}x}{\mathrm{d}y}=\frac{1}{\frac{\mathrm{d}y}{\mathrm{d}x}}=\frac{x}{f'(\ln x)}$.
    \[
    \begin{aligned}
        \frac{\mathrm{d}^2 x}{\mathrm{d}y^2}=&\frac{\mathrm{d}}{\mathrm{d}y}\left(\frac{x}{f'(\ln x)}\right)=\frac{\mathrm{d}}{\mathrm{d}x}\left(\frac{x}{f'(\ln x)}\right)\cdot\frac{\mathrm{d}x}{\mathrm{d}y}\\
        =&\frac{f'(\ln x)-x\cdot\left(f''(\ln x)\cdot\frac{1}{x}\right)}{(f'(\ln x))^2}\cdot\frac{x}{f'(\ln x)}\\
        =&\frac{x(f'(\ln x)-f''(\ln x))}{(f'(\ln x))^3}.
    \end{aligned}
    \]

    \textbf{法二.} 对$y=f(\ln x)$两边关于$y$求导得
    \[
    1=f'(\ln x)\cdot\frac{1}{x}\cdot\frac{\mathrm{d}x}{\mathrm{d}y},\qquad\tag{*}
    \]

    解得$\frac{\mathrm{d}x}{\mathrm{d}y}=\frac{x}{f'(\ln x)}.$

    式(*)变形得$x=f'(\ln x)\cdot\frac{\mathrm{d}x}{\mathrm{d}y}$,两边关于$y$再求导得
    \[
    \frac{\mathrm{d}x}{\mathrm{d}y}=f'(\ln x)\frac{\mathrm{d}^2 x}{\mathrm{d}y^2}+f''(\ln x)\cdot\frac{1}{x}\cdot\left(\frac{\mathrm{d}x}{\mathrm{d}y}\right)^2,
    \]

    解得
    \[
    \frac{\mathrm{d}^2 x}{\mathrm{d}y^2}=\frac{\frac{\mathrm{d}x}{\mathrm{d}y}-\frac{f''(\ln x)}{x}\left(\frac{\mathrm{d}x}{\mathrm{d}y}\right)^2}{f'(\ln x)}=\frac{x(f'(\ln x)-f''(\ln x))}{(f'(\ln x))^3}.
    \]

    \item[15.] \textbf{Solution}. $x_{n+1}-x_n=\frac{1}{n+1}-\ln(n+1)+\ln n=\frac{1}{n+1}-\ln\left(1+\frac{1}{n}\right)<0$,
    
    因此数列$\{x_n\}$单调减少.
    \[
    \begin{gathered}
         x_n=1+\frac{1}{2}+\cdots+\frac{1}{n}-\ln n>\ln\left(1+\frac{1}{1}\right)+\ln\left(1+\frac{1}{2}\right)+\cdots+\ln\left(1+\frac{1}{n}\right)-\ln n\\
        =\ln(n+1)-\ln n>0,
    \end{gathered}
    \]

    即数列$\{x_n\}$有下界0,由单调有界收敛准则知数列$\{x_n\}$收敛.
\end{enumerate}

\section{证明题(每小题 5 分,共 10 分)}
\begin{enumerate}
    \item[16.] \textbf{Proof}.设$f(x)=\frac{\sin x}{x}-\alpha\cos x$,则$f(x)$在$\left(0,\frac{\pi}{2}\right]$上连续.
    
    因为$f(0^+)=1-\alpha<0$,所以$\exists\,x_1\in(0,\delta)$($\delta$是足够小的正数),$f(x_1)<0$.

    而$f\left(\frac{\pi}{2}\right)=\frac{2}{\pi}>0$,显然$f(x)$在$\left[x_1,\frac{\pi}{2}\right]$上连续,由零点定理知存在$\xi\in\left(x_1,\frac{\pi}{2}\right)$,
    
    使得$f(\xi)=0$即$\alpha\xi=\tan\xi$,故方程$\alpha x=\tan x$在$\left(0,\frac{\pi}{2}\right)$内至少有一个实根.

    \item[17.] \textbf{Proof}. 令$F(x)=f(x)-x$,则$F(x)$在$[0,1]$上连续,在$(0,1)$内可导,且$F(\theta)F(1)=-(1-\theta)<0$,
    
    由零点定理可得,存在$c\in(\theta,1)$,使得$F(c)=0$.

    再令$G(x)=\mathrm{e}^{-x}F(x)$,因为$G(0)=G(c)=0$,由Rolle定理得存在$\xi\in(0,c)\subset(0,1)$,使得
    \[
    G'(\xi)=0.
    \]

    注意$G'(x)=\mathrm{e}^{-x}(f'(x)-1-f(x)+x)$,有$f'(\xi)-f(\xi)+\xi=1$.
\end{enumerate}
\end{document}
